
\documentclass[a4paper]{article}

%% Language and font encodings
\usepackage[english,spanish]{babel}
\usepackage[utf8x]{inputenc}
\usepackage[T1]{fontenc}

\usepackage{lipsum}

%\bibliographystyle{apalike} %% Sets page size and margins
\usepackage[round]{natbib}
\bibliographystyle{plainnat}
%\usepackage[a4paper,top=3cm,bottom=2cm,left=3cm,right=3cm,marginparwidth=1.75cm]{geometry}

%% Useful packages
\usepackage{amsmath}
\usepackage{graphicx}
\usepackage[colorinlistoftodos]{todonotes}
\usepackage[colorlinks=true, allcolors=blue]{hyperref}
\usepackage{subfigure} 
\usepackage{float}
\usepackage{verbatim}
\usepackage{appendix}

\addtolength{\subfigcapskip}{-10pt}
\addtolength{\subfigbottomskip}{-30pt}
%\addtolength{\subfigcapmargin}{-30pt}


%\addtolength{\subfigtopskip}{-25pt}

\graphicspath{{../plots/}}

\usepackage{listings}
\usepackage{xcolor}


\lstset{language=R,
	basicstyle=\small\ttfamily,
	stringstyle=\color{DarkGreen},
	otherkeywords={0,1,2,3,4,5,6,7,8,9},
	morekeywords={TRUE,FALSE},
	deletekeywords={data,frame,length,as,character},
	keywordstyle=\color{blue},
	commentstyle=\color{DarkGreen},
}



\renewenvironment{abstract}{
	\vspace*{\fill}
	\begin{center}%
		\bfseries\abstractname
\end{center}}%
{\vfill}


%% authors
\usepackage{authblk}
\title{Old series, new signals \\
	 {\large The economic cycle in light of wavelet analysis} }
\author[1]{Diego Kozlowski}
\affil[1]{Master in Data Mining \& Knowledge Discovery, FCEN-UBA \\ diegokoz92@gmail.com}

\date{}                     %% if you don't need date to appear
\setcounter{Maxaffil}{0}
\renewcommand\Affilfont{\itshape\small}


\begin{document}


\maketitle


\selectlanguage{english}
	\begin{abstract}
	The economic cycle is a subject of recurring debate in the specialized bibliography. Both from the conceptual point of view and its empirical recognition, there is no general consensus regarding the causes and concrete forms of this characteristic of the economy. In particular, the statements regarding the existence of a long wave, proposed by different authors of the early twentieth century, are a source of debate. In this work we propose an empirical review of series traditionally used in economic analysis, such as the product, wages and gold, for the United States and the United Kingdom, from the 18th century to the present, using a technique originally developed in the area of ​​signal analysis. The objective is to seek new evidence regarding the presence of a well-defined cycle in long periods of economic history. For this, the Wavelets analysis is used in search of low frequency signals, or long periods. The results show favorable evidence to the existence of three well-defined cycles, one of which would be of an amplitude around 50 years.
	\end{abstract}

\selectlanguage{spanish}	
	\begin{abstract}
		El desenvolvimiento cíclico de la economía es un tema de recurrente debate en la bibliografía especializada. Tanto desde el punto de vista conceptual como en el reconocimiento empírico, no existe un consenso generalizado respecto de las causas y formas concretas de esta característica de la economía. En particular, son fuente de debate las afirmaciones respecto a la existencia de ciclos definidos en períodos largos, propuestos por diferentes autores de principios del siglo XX. En el presente trabajo se propone una revisión empírica de series tradicionalmente utilizadas en el análisis económico, como lo son el producto, salario y oro, para Estados Unidos y el Reino Unido, desde el siglo XVIII a la actualidad, mediante una técnica originalmente desarrollada para el área de análisis de señales. El objetivo es buscar nuevas evidencias respecto a la presencia de un ciclo bien definido en períodos largos de la historia económica. Para esto, se utiliza el análisis de Wavelets en busca de señales de baja frecuencia, o períodos largos. Los resultados muestran evidencia favorable a la existencia de tres ciclos bien definidos, uno de los cuales sería de una amplitud en torno a los 50 años.
	\end{abstract}

\selectlanguage{english}

\section{Introduction}
 
In economic theory there is no unequivocal position regarding the cyclical forms of economic development. The long waves proposed by \cite{kondratieff1979long} of approximately 50 years span and the medium waves proposed by \cite{kuznets1930secular} have result in controversy throughout the 20th century. To a large extent this debate is due to the fact that there is no unambiguous expression of what is called \textit{economic development}. In the first place, it doesn't exist a single variable that captures this concept as a hole, but it can only be represented fragmentarily in variables such as the Gross Domestic Product (GDP), the Wages, or the Interest Rate. But even if we could have a single variable that completely measures the economic development, the national delimitation of the measurements is still remaining. The latter comes from the fact that what is considered as the economic cycle refers to a fundamental characteristic of the economic system, which is not necessarily mediated by national divisions. That is to say, it is a phenomenon proper to capitalism as a system, and does not necessarily reproduce itself fully within each country.


These complexities for measuring the economic cycle make it difficult to understand it. The goal of this paper is to make use of new quantitative tools for the analysis of time series, to review the empirical evidence regarding the existence of the economic cycle. Given the limitations mentioned above, it was decided to use series from the United States because it is a country that due to its size achieves, from the 20th century, to represent, at least partially, the general tendencies of the economy. For methodological reasons, it is necessary that the information used goes back to the 19th century, a century in which it is not clear that the general characteristics of economic development are expressed within this country. That is why for the eighteenth and nineteenth centuries the study is complemented with the product series for the United Kingdom.

The paper is structured as follows: After this introduction, a summary of the main debates regarding the economic cycle that took place during the 20th century is presented. In the third section an exploratory data analysis is carried out, where the different series used in the rest of the work are observed, as well as the characteristics of the gold price series and its effects on the rest of the data. Finally some of the series along with the known crises in the economic historiography for the United States in the 20th century are presented. The fourth section proposes the use of the wavelet technique to model the economic cycle and we observe the results of applying this methodology to the GDP and wages series. Finally, the fifth and final section presents the conclusions and future lines of work.


\section{Debates around the economic cycle}

There are not many polemics in the economic literature in which position has been taken from practically all the schools of economic thought. The discussion around what is the economic cycle? How does it originate? And what implications does it have at the level of economic policy, is for sure one of those controversies.

From all schools of economic thought it have been sought explanations to these questions. On the one hand, we find the explanations that focus on the role of demand, particularly investment goods, as the starting point of the economic cycle. There, in the work of \cite{kalecki2013essays} the cycle arises by the particular dynamics of the demand for investment goods from the temporary differences between the decisions of demand for these goods and the moment when they are finally set in motion. Then \cite{keynes2018general} argues that \textit{animal spirits} dominate the scene, with the marginal efficiency of capital guiding the cycle path. Also belong to this school \cite {harrod1936trade}, \cite{kaldor1940model} and \cite{samuelson1939synthesis}, who propose models where there is an interaction between the Keynesian multiplier and the acceleration principle. That is, where the product defines the demand for consumption goods, and this determines the demand for investment goods, which then operate on the product, generating a spiral of over-determinations that end up producing a cycle.

From a different school of thought \cite{schumpeter1939business} begins with the role of innovative entrepreneurs to reach a "tricyclic model" where a superposition of short, medium and long waves operates.
The Austrian theory of the cycle, headed by \cite{hayek1933} and \cite{von1943elastic}, considers a purely monetary origin, based on the endogenous creation of purchasing power and changes in relative prices.

Finally, there is the neoclassical theory of the cycle, which emerges from Lucas's critique of Keynesian macroeconomics. In this kind of models, the microfundation of behavioral assumptions is prioritized. This means that individuals are rational agents, and that at all times there are competitive equilibriums. These models are divided between those proposed by \cite{lucas1975equilibrium} where the initial shock is monetary, and the models of the real business cycle \citep{plosser1989understanding} where the original impulse is given by random changes in technology.

From the point of view of the empirical analysis, there are multiple authors who have found evidence, either through the detailed study of different series \citep{kuznets1930secular, kondratieff1979long, schumpeter1939business}; or using autoregressive models, of moving averages and ARIMA \citep {hamilton1989new, kaiser2012measuring}; or more recently by using wavelets \citep {yogo2008measuring, soares2011business}. Notwithstanding the latter, the use of new empirical techniques for cycle analysis remains a fertile ground for research.

\section{Exploratory Data Analysis}\label{EDA}

In the present section we will carry out a brief exploratory analysis of data to observe the general characteristics of the series.

\subsection{Information sources}

Given that the objective of this paper is to perform an analysis of the economic cycle that takes Kondratieff's long waves into consideration, it is necessary to have information as widespread as possible, and therefore use different sources.

For the United States GDP, the data from 1929 to the present comes from the information provided by the \textit{Bureau of Economic Analysis} of that country. For the data from 1790 to 1929 the series elaborated by \cite{johnston2018us} was used.

For the annual series of the nominal hourly wage of production workers in the United States, the information found in \cite{officer2009two} and complemented by \cite{Roesch2018} was used. Nowadays this series corresponds to the item \textit{Employer Costs for Employee Compensation, Total Compensation, Manufacturing, Private Industry} from the series of the \textit{Bureau of Labor Statistics} of the United States.

The annual series of gold prices in the New York market between 1791 and 2017 is based on \cite{officer2018gold}

To analyze the period preceding 1900 the UK GDP series expressed in gold was used. Given the changes in the geopolitical boundaries of the United Kingdom, it was decided to use a series that was consistent intertemporally, the objective being to recognize fluctuations that do not depend on changes in the registration limits. For this, the nominal GDP series between 1700 and 1900 of \cite{Williamson2018uk} was used. The price of gold in the London market for the period 1718-1900 and the official British price for the period 1700-1718 were both recovered from \cite{officer2018gold}


\subsection{Original series}

The figure \ref{fig:oro} shows the value of an ounce of gold in nominal dollars in the New York market, between 1791 and 2017. There, the exit from the gold standard of the world economy is marked in 1971. After the second world war and until that year, the global monetary structure was based on parity with the dollar and the \textit{nominal anchor} of the same with the gold reserves of the Federal Reserve Board (FED). This means that the United States could not issue dollars that were not backed by their gold equivalent in the reserves of the central bank of that country. Therefore, the dollar-gold ratio remained practically unchanged until 1971. After eliminating this anchor, the capacity of free issuance without backup in gold allowed the FED to issue above the reserves it had, and in general, over gold in circulation. This has led to an increase in the price of gold expressed in dollars, or equivalently, a fall in the US dollar expressed in gold.

\begin{figure}[H]
	\centering
	\includegraphics[width=0.75\linewidth]{oro_en.png}
	 \caption{}\label{fig:oro}
\end{figure}

Given that what is sought in this paper is a long-term analysis of the economic cycle, this nominal disturbance obscures the underlying phenomenon that is being searched. That is why we choose to normalize the nominal GDP series and the nominal wage by the price of gold. In this sense, the series is read as the product and wage expressed in its capacity to purchase gold.


Since gold is a refuge of value in crisis periods, its price is counter-cyclical, a difference from what happens with the Consumer Price Index (CPI). Therefore normalizing by gold instead of CPI gives a better comprehension of the cycle in the series studied and the empirical analysis is facilitated.

According to the above, the figure \ref{fig:PBI} shows the series of the United States GDP between 1900 and 2017, expressed in gold. Meanwhile the figure \ref{fig:salary} shows the series of the hourly wage of a worker of production in the United States between 1900 and 2017, also expressed in gold.

In both cases the periods of economic distress known by the literature are highlighted in red, and in punctuated lines those punctual crises that occurred in a particular year. The table \ref{tabla_crisis} marks the detail of these.


\begin{figure}[H]
	\centering
	\subfigure[]{\includegraphics[width=0.75\linewidth]{gdp_in_gold_eda_en.PNG}
	\label{fig:PBI}}
	\subfigure[]{\includegraphics[width=0.75\linewidth]{wg_in_gold_eda_en.PNG}
	\label{fig:salario}}
	\caption{GDP and Wage. US. Crisis highlighted} \label{fig:series_crisis}
\end{figure}


\begin{table}[ht]
	\centering
	\begin{tabular}{ll}
		\hline
		Period & Crisis \\ 
		\hline
		1907 & Panic of 1907 \\ 
		1920 – 1921 & Depression of 1920 – 21 \\ 
		1929 – 1939 & Great Depression \\ 
		1970s & 1970s Energy Crisis \\ 
		1973 & OPEC Oil Price Shock \\ 
		1979 & Iranian Revolution\\ 
		1980s & Early 1980s Recession\\ 
		1987 & Black Monday \\ 
		1990s & Early 1990s Recession\\ 
		2000 & Dot-com Bubble \\ 
		2001 & 911 \\ 
		2008 & Subprime Global Financial Crisis \\ 
		\hline
	\end{tabular}
\caption{Main US and Global Crisis during 20th century.}
\label{tabla_crisis}
\end{table}

In the first place what is observed is the similarity of both series. They Both show three peaks, during the 20', in 1970 and 2000, followed by deep falls. The normalization by the gold-price allows to see a great cycle with three oscillations, at least apparently, during the twentieth century. The crises reviewed by the economic history literature seem to have their correlate in the movements observed in both series. Besides, it is also interesting to note that the third upward movement, whose peak is in the year 2000, leads to a value similar to that of the previous oscillatory movement for the case of GDP, but not for wages. This expresses that the distribution of the GDP between wage and profit has changed in the last period.

To complement the analysis of the United States series, it is interesting to observe the movement of the product at the United Kingdom for the preceding centenaries. During the eighteenth and nineteenth centuries this country was parapet as the core of global accumulation, and therefore it may be possible to find evidence of the economic cycle in this country in particular. The figure \ref{fig:uk_gdp} shows the GDP of the United Kingdom, between 1700 and 1900. It is normalized by the price of gold in the London market from 1718, while the first years correspond to the official British price. At the same time, the years of fall of the GDP from 1800 are stand out, given that for the eighteenth century there are no visually outstanding points.

\begin{figure}[H]
	\centering
	\includegraphics[width=0.75\linewidth]{uk_gdp_en.png}
	\caption{} 
	\label{fig:uk_gdp}
\end{figure}


What is observed is a softer upward movement than that seen in the 20th century. It is worth mentioning that the movement of the product in the United Kingdom during the twentieth century retains an important similarity with that seen in the case of the United States. In the figure \ref{fig:uk_gdp}, however, a cyclic movement is observed of around 10 years of extension. During the nineteenth century it can be noted that every 7-11 years there is a fall in the product in terms of its ability to buy gold. Just as the twentieth century accounts for three large oscillations, the nineteenth century clearly marks the shorter oscillations, around 10 years, while the eighteenth century in relative terms expresses greater stability.

In the following section, the described series will be used as inputs for a technique coming from the field of signal processing, Wavelets, which allows to highlight automatically the most important cyclic amplitudes of the series.

\section{Wavelets}

Although in econometrics the use of autoregressive models and moving averages for the analysis of time series is extended, there are other widely disseminated techniques in the literature of signal analysis that are not yet extensively used in the study of economic series. A classic example in this sense are the Fourier series. This technique analyze how any time series can be thought of as a composition of periodic functions, that is, as the sum of sines and cosines. In this way, a space of \textit{frequency} ($1/period$) is constructed where the periodic functions are defined according to their frequency (or extension in time, horizontal movement) and amplitude (vertical movement). Figure \ref{fig:ciclo} exemplify these definitions.


\begin{figure}[H]
	\centering
	\includegraphics[width=0.65\linewidth]{ciclo_en.png}
	\caption{period and amplitude} \label{fig:ciclo}
\end{figure}

The Fourier decomposition is based on a transformation of domain of series, from the time domain, to the frequency domain.
\\

Wavelets are also a type of transformation on the original series that can be thought of as a rotation of a space of functions to a different domain. But unlike the Fourier transform that has sines and cosines as the basis of functions, the Wavelet transform has a particular type of base function, called \textit{Wavelets} \cite{castro1995wavelets}. A base of this type is constructed from a mother function, which is a short wave, of finite duration. That is to say, unlike the sine and cosine functions that extend infinitely across the domain, the wavelets have \textit{compact support}, this means that they use as a base functions that do not extend infinitely in the time domain. Another characteristic of wavelet functions is that the area under the curve must be zero, they are centered at zero. This mother function moves and dilates to build an orthonormal basis.

Whereas a Fourier transformation of a series from the time domain to the domain of the \textit{frequency} takes the form of:

$$
X(F)=\int_{-\infty}^{\infty} x(t) e^{-j2\pi Ft}dt
$$

The wavelet transform goes from the domain of time to the domain of \textit{scale} and \textit{translation}:

$$
X(a,b)=\int_{-\infty}^{\infty} x(t) \psi^*_{a,b}(t)dt
$$

The scale describes the frequency (inverse of the extension or period) of the cycle, while the translation describes the movement along the series. Since low-frequency series (longer cycles) occupy a larger portion of the series, they are harder to locate at a particular moment in time. For the above, the resolution in low frequencies is bad in the time domain, but good in the frequency domain, while the high frequency cycles have high resolution in the time domain, but less resolution in the domain frequency domain. If we compare it with the Fourier transforms, we might think that it has very high resolution in the frequency domain, but no resolution in the time domain. The wavelet manages to define the presence of a certain frequency in a certain moment of time.

It is also possible to understand wavelets as a correlation  analysis between a time series, and some compact wave function at one point of time and a certain frequency. The translations generate a shift of the wave function, and therefore we can calculate the correlation for the entire time range. Rescaling modifies the frequency of the wave function, which allows to calculate the correlation for several different wave frequencies. The amount of available data is what defines the ability to rescale the base function. That is, how low are the minimum frequencies that can be analyzed. Finally, what we obtain is a value of the linear association between the original series and the base function, for each value of time and frequency.

The base function we use for the present work is called \textit{Morlet Wavelet}, which as implemented in the WaveletComp \citep{Roesch2018} library has the following functional form:

$$
\psi(t)=\pi^{-\frac{1}{4}}e^{i\omega t}e^{\frac{-t^2}{2}}
$$


Where $\omega$ is the angular frequency (rotation rate in radians per unit of time). This is a continuous, complex function, frequently used in the literature \citep{conraria2011continuous}. The base from the translation, $a$, and the scaling, $b$, implemented is:

$$
X(a,b)=\sum_{t} x(t)   \frac{1}{b} \psi^*\left(\frac{t-a}{b}\right)dt
$$

Visually, the translations and rescaling of the base function can be seen in the figure \ref{fig:morlet}. There is seen that the translations are defined in the time domain, while rescaled do in the frequency domain. If the time-frequency plane is reconstructed and the correlation of each point with the original series is calculated, a new dimension is obtained that represents the degree of adjustment of our series at each frequency, for the different moments of time.

\begin{figure}[H]
	\centering
	\includegraphics[width=\linewidth]{morelt_en.png}
	\caption{Translation and rescaling of the Morlet base function} \label{fig:morlet}
\end{figure}

Finally, the results can be displayed on a \textit{spectrogram}, ie, it can be seen for each frequency at each moment of time, the degree of correlation of the original series with Morlett function.


To visualize the wavelets in the analysis of the economic cycle it is useful to define a theoretical model of a cyclical economy, with the different components seen separately and in their composition. This way we can observe the characteristics of the spectrogram in this model and then compare it with the real data.

In the figure \ref{fig:serie_teorica} 100 values of the different components with which we will build the series are observed, they are:


\begin{itemize}
	
	\item \textbf{impulse}: A series with a constant value of 50 in which a particular period takes the value 100.
	\item \textbf{Trend}: Grows half a point per period.
	\item \textbf{short cycle}: A cycle of small amplitude and extension
	\item \textbf{middle cycle}: A cycle of medium amplitude and extension
	\item \textbf{long cycle}: A cycle of large amplitude and extension
	\item \textbf{noise}: Noise generated from a normal distribution, not significant with respect to the amplitude of the cycles and the slope of the trend.
\end{itemize}

the R code which generates those series is as follows:

\begin{lstlisting}
n = 1000
impulse= c(rep(50,(n/2-1)),100,rep(50,n/2))
trend = c(1:n)/2
short = 10 * sin(( 2 * pi/3 ) * c(1:n))
middle = 20 * sin(( 2 * pi/10) * c(1:n))
long = 30 * sin(( 2 * pi/50) * c(1:n))
noise <- rnorm(n)
composite_series = impulse + trend + short + middle + long + noise
\end{lstlisting}


\begin{figure}[H]
	\centering
	\includegraphics[width=\linewidth]{serie_teorica_en.PNG}
	\caption{} \label{fig:serie_teorica}
\end{figure}

In the figure \ref{fig: espect_teo} the spectogram of each one of the mentioned elements is observed. This plots shows on the vertical axis the period, the inverse of the wave frequency, which corresponds to the distance between the valleys or peaks of a cycle. The color scale (Blue for the lowest values to red in the highest values) represents the amplitude of the cycle. the horizontal axis represents the calendar time. That is, for each calendar time we can observe the amplitude of the cycle in each of the possible wave frequencies.


\begin{figure}[H]
	\centering
	\subfigure[impulse]{\includegraphics[width=0.49\linewidth]{espectograma_teorico_impulso_en.png}}
	    \vspace{0.00mm}
	\subfigure[trend]{\includegraphics[width=0.49\linewidth]{espectograma_teorico_tendencia_en.png}}
	    \vspace{0.00mm}
	\subfigure[3 year cycle]{\includegraphics[width=0.49\linewidth]{espectograma_teorico_ciclo_3_en.png}}
	    \vspace{0.00mm}
	\subfigure[10 year cycle]{\includegraphics[width=0.49\linewidth]{espectograma_teorico_ciclo_10_en.png}}
	    \vspace{0.00mm}
	\subfigure[50 year cycle]{\includegraphics[width=0.49\linewidth]{espectograma_teorico_ciclo_50_en.png}}
	    \vspace{0.00mm}
	\subfigure[normal noise]{\includegraphics[width=0.49\linewidth]{espectograma_teorico_ruido_en.png}}
	    \vspace{0.00mm}
	\subfigure[composite series]{\includegraphics[width=0.75\linewidth]{espectograma_teorico_composicion_series_en.png}}
	\caption{Theoretical spectrograms} \label{fig:espect_teo}
\end{figure}

As shown in the figure \ref{fig:espect_teo} the impulse is presented as a cycle of large amplitude in all frequencies for the moment corresponding to the jump. The tendency, on the other hand, is basically shown as noise, given that it is strictly a non-cyclical behavior. Nonetheless it has the characteristic of taking higher amplitude values towards the end of the period for low frequencies, and particularly low amplitude values in the high frequencies of the first moments.

The three defined cycles clearly mark a horizontal line in the corresponding period, which is then diluted towards the other frequencies. Finally, normal noise has a very particular behavior, showing greater amplitudes, irregularly, at high frequencies, and homogenizing towards a low amplitude value in longer cycles. This is due to the fact that normal noise can resemble a cycle of very short periods due to a succession of ups and downs, but since it is a random process, it is increasingly unlikely to resemble cycles of longer periods, being that this would imply a greater number of successions of consecutive increases and decreases. In turn, both in the spectrogram of normal noise and in that of the trend, it can be seen how the graph loses resolution for longer periods, as mentioned above.


Finally, in the composition of series it is observed how the cycles of greater amplitude and frequency are expressed in the chromatic scale in a sharper way than the cycles of smaller amplitude and frequency. It is important to note that the agreement between amplitude and frequency is a product of the way in which we build the series, since we expect that the longer economic cycles will also correspond to movements of greater amplitude.

It is worth mentioning that the choice for the theoretical model of these three levels and cyclical amplitudes is not arbitrary, but corresponds roughly to what is considered by the literature: \cite{kondratieff1979long} studies the long series, about 50 years, while \cite {kuznets1930secular} proposes secular movements between 15 and 25 years. Finally the Real business cycle \citep{kydland1982time} considers a short cycle.

With the analysis from the figure \ref{fig:espect_teo} we can now see the results of the original series. In the figure \ref{fig:espect_PBI_a} the spectrogram corresponding to the GDP of the United States expressed in gold can be observed. There the difference in the series is clearly marked before and after 1900, and in particular the studied break of the 70's is also shown. Nevertheless, for that period there are roughly 3 frequencies where a cyclical behavior is registered. In the approximate periods of 8 and 50 years, and a differentiated cycle of the latter, of approximately 30 years. Given the heteroscedasticity of the series, in \ref{fig:espect_PBI_b} the spectogram of the same series taken in logarithm of base 10 is proposed. Here the cycle of 50 years extends beyond the time, until the middle of the 19th century. For its part, a shorter cycle of approximately three years appears briefly in the 70s.

\begin{figure}[H]
	\centering
	\subfigure[]{\includegraphics[width=0.75\linewidth]{espectograma_gdp_en.png}
	\label{fig:espect_PBI_a}}
	\subfigure[]{\includegraphics[width=0.75\linewidth]{espectograma_log_gdp_en.png}
	\label{fig:espect_PBI_b}}
	\caption{US GDP in gold. 1790-2017} \label{fig:espect_PBI}
\end{figure}

the figure \ref{fig: espect_wg} show the spectrograms of the wage series expressed in gold in \ref{fig: espect_wg_a} and the same taken in log base in \ref{fig: espect_wg_b}. For this series we can again observe a well-defined long cycle around 50 years, especially if we look at the series taken on a logarithmic basis. This long cycle seems to oscillate between the frequencies of $1/32$ and $1/64$, falling in time. On the other hand, it delimits a second cycle of around 16 years of extension, and finally a short cycle of between 6 and 8 years. When taking the logistic scale, also appears a cycle of higher frequency of about 3 years of extension. Consequently, both series seem to yield results in agreement, whether or not they are taken in logarithm.


\begin{figure}[H]
	\centering
	\subfigure[]{\includegraphics[width=0.75\linewidth]{espectograma_wg_en.png}
	\label{fig:espect_wg_a}}
	\subfigure[]{\includegraphics[width=0.75\linewidth]{espectograma_log_wg_en.png}
	\label{fig:espect_wg_b}}
	\caption{US wage in gold. 1790-2017} \label{fig:espect_wg}
\end{figure}

The figure \ref{fig:espect_uk} shows the GDP series expressed in gold for the United Kingdom between 1700 and 1900. As in the previous series, it is expressed without additional transformations in \ref{fig: espect_uk_a} and in log base in \ref{fig:espect_uk_b}. Unlike the US series, here we can see more clearly the short cycle, around 10 years.

\begin{figure}[H]
	\centering
	\subfigure[]{\includegraphics[width=0.75\linewidth]{espectograma_gdp_uk_en.png}
		\label{fig:espect_uk_a}}
	\subfigure[]{\includegraphics[width=0.75\linewidth]{espectograma_log_gdp_uk_en.png}
		\label{fig:espect_uk_b}}
	\caption{UK GDP in gold. 1700-1900} \label{fig:espect_uk}
\end{figure}

Besides, another difference that stands out is that in the plots for UK the effect of the logarithmic transformation is greater than in the previous series. Thus, in \ref {fig:espect_uk_b} we can observe a multiplicity of cycles that stand out as influential. In particular, the shortest one seems to be around the four years of duration, especially present at the beginning of the series and near the 1800. Also highlights the cycle around the 8 years of duration, associated in the description made in Exploratory Data Analysis with what was marked as a 10

-year cycle. Of greater temporal extension a cycle is noted around the 20 years, at the beginning of the series and in the environment of 1800, together with a cycle of 30-35 years in the remaining periods. Finally, a long cycle is again recorded, around 64 years of extension.

It is important to record the effect produced by the trend of the series on the spectrograms. In section \ref{EDA} the figure \ref{fig:uk_gdp} showed a strong trend effect of the product expressed in gold for the United Kingdom during the eighteenth and nineteenth centuries. This trend did not seem to be linear and therefore it is of our interest to analyze the possible distortions that this can generate on the spectrograms. in the figure \ref{fig:tendencias} we can see the GDP of the United Kingdom in gold, together with a linear smoothing using a local regressions model \citep{Shyu1992}, and the series resulting from eliminating this trend.

In fact, the figure \ref{fig:espectograma_gdp_uk_Tend} shows the spectrogram of the trend calculated on the series. There stands out the period around fifty years, which seems to indicate that what was described before is a simple trend effect hidden in the spectrograms. However, in the figure \ref{fig:espectograma_sin_tend} shows the spectrogram of the series of UK GDP, also expressed in gold but after subtracting the trend. This spectogram is almost identical to that seen previously in the figure \ref{fig:espect_uk_a}. This means that the wavelet technique is not affected by the underlying trend of the series analyzed.

\begin{figure}[H]
	\centering
	\subfigure[]{\includegraphics[width=0.75\linewidth]{pbi_uk_tendencias_en.png}
		\label{fig:tendencias}}
	\subfigure[]{\includegraphics[width=0.48\linewidth]{espectograma_gdp_uk_Tend_en.png}
		\label{fig:espectograma_gdp_uk_Tend}}
	\subfigure[]{\includegraphics[width=0.48\linewidth]{espectograma_gdp_uk_sinTend_en.png}
		\label{fig:espectograma_sin_tend}}
	\caption{Trend effect} \label{fig:espect_tendencias}

\end{figure}

\section{Conclusions}

In this work we made a tour trough the main theoretical approaches with respect to the economic cycle, and highlighted the importance of an empirical analysis. For this latter, we use the GDP and wages of the US between 1790 and 2017, expressed in gold. We also looked at the GDP series for the UK, between 1700 and 1900.

In the Exploratory Data Analysis we found a strong correspondence between the breaks of these series and the crises known by the economic historiography. Also, we marked an apparent oscillatory movement that seemed to correspond with the long waves of Kondratieff. For its part, the nineteenth century in the United Kingdom seemed to highlight the crises around the 10 years of duration.

Then we explore wavelets-analysis, a tool with which its possible to visualize the correspondence of economic series with cycles of different extensions. The results of this technique showed the existence of three well-defined cycles at different frequencies, which correspond to the hypotheses studied on the existence of a short cycle, a medium cycle and a long cycle. It is also important to mention that this tool loses resolution in cycles of very long periods. In this sense, the tool seen does not allow to define exactly the temporal extension of each one of the cycles, but it does provide evidence of their existence.

It is important to note that the objective of this paper is to look for empirical evidence regarding the frequency and amplitude of the cyclical behavior of the economy. The wages and GDP series are taken because they are good approximations of the general economic movement, but they are not the only ones. In turn, statistics have a national base, the GDP is always from a particular country, as well as wages statistics, and so we needed to chose a particular country as expression of the world economy. In this sense, since the United States is the largest national unit of the world's economy during the 20th century, we begin our analysis with its series. However, while the US economy is a good reflection of the movements of the world economy for the twentieth century, the same does not stand for the nineteenth century. In this sense, it is natural that the general determinations of the economy, such as the cycle, are not fully expressed for that century, and we observe evidence only from 1900 onwards. That is why we complemented the analysis with the GDP of UK for the two preceding centuries.

To conclude, the use of this technique for the study of historical series, widely analyzed by specialized bibliography, seems to be useful to obtain new information from the data used. The results point to the confirmation of the existence of series of medium and long duration, although the empirical regularity is not sufficient to determine with accuracy the frequency of this cycles. In other words, although the intuition of \cite{kuznets1930secular} and \cite{kondratieff1979long} is confirmed, the results are less promising regarding the possibility of precisely defining the phenomena. Even more, the evidence seems to indicate that the frequency of medium and long duration cycles could vary partially throughout history.

The present work proposes multiple lines of research, especially regarding the use of Wavelets technique in new series, both for series of financial variables, as well as GDP and wages series for other countries.

\bibliography{bibliography.bib}



\end{document}
